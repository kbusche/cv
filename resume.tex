% resume.tex
% vim:set ft=tex spell:
% template adapted from:
%    https://www.toofishes.net/blog/why-i-do-my-resume-latex/


\documentclass[10pt, letterpaper]{article}
\usepackage[letterpaper,margin=0.75in]{geometry}
\usepackage[utf8]{inputenc}
\usepackage{mdwlist}
\usepackage[T1]{fontenc}
\usepackage{textcomp}
\usepackage[normalem]{ulem}
\usepackage{verbatim} % for multiline comments
\usepackage{tgpagella} % serif font
\usepackage{cabin} % sans serif font 
\pagestyle{empty}
\setlength{\tabcolsep}{0em}
\setlength{\parindent}{0em}

% indentsection style, used for sections that aren't already in lists
% that need indentation to the level of all text in the document
\newenvironment{indentsection}[1]%
{\begin{list}{}%
	{\setlength{\leftmargin}{#1}}%
	\item[]%
}
{\end{list}}

% opposite of above; bump a section back toward the left margin
\newenvironment{unindentsection}[1]%
{\begin{list}{}%
	{\setlength{\leftmargin}{-0.5#1}}%
	\item[]%
}
{\end{list}}

% format two pieces of text, one left aligned and one right aligned 
\newcommand{\headerrow}[2]
{\begin{tabular*}{\linewidth}{l@{\extracolsep{\fill}}r}
	#1 &
	#2 \\
\end{tabular*}}

% make "C++" look pretty when used in text by touching up the plus signs
\newcommand{\CPP}
{C\nolinebreak[4]\hspace{-.05em}\raisebox{.22ex}{\footnotesize\bf ++}}

\newcommand{\sansserif}{\cabin}


% actual content starts here
\begin{document}
{\sansserif \LARGE \textbf{Kirk R. Busche}}

\rule{\textwidth}{0.5ex}
\vspace{-2em}
\begin{center}
	{\sansserif\small \textit{kbusche2@illinois.edu $\bullet$ 507-261-4380}}
\end{center}

%% --- EDUCATION --- %%
\subsection*{\sansserif EDUCATION:}

\begin{comment}
%% Ph.D student %%
\headerrow
	{\textbf{Ph.D. Student in Electrical and Computer Engineering}}
	{\textbf{August 2016 - Present}}

University of Illinois Urbana-Champaign

College of Engineering

GPA: 3.94

\newlength{\mylength}
\settowidth{\mylength}{Relevant Coursework:}
\hangindent=\mylength
Relevant Coursework: Pattern Recognition, Digital Signal Processing II, Vector
Space Signal Processing, Topics in Image Processing, Random Processes,
Optimization for Computer Vision, Digital Imaging
%% TODO fix hangindent issues
\hangindent=0ex
\\
\end{comment}

%% Masters %%
\headerrow
	{\textbf{Master of Science in Electrical and Computer Engineering}}
	{\textbf{August 2016 - December 2019}}

University of Illinois Urbana-Champaign

College of Engineering

GPA: 3.94

\newlength{\mylength}
\settowidth{\mylength}{Relevant Coursework:}
\hangindent=\mylength
{Thesis: \textit{Frequency-Modulated Continuous-Wave Radar Processing
	Fundamentals}}

Relevant Coursework: Pattern Recognition, Digital Signal Processing II, Vector Space Signal Processing, Topics in Image Processing, Random Processes, Optimization for Computer Vision, Digital Imaging
%% TODO fix hangindent issues
\hangindent=0ex
\\

%% Bachelors %%
\headerrow
	{\textbf{Bachelor of Electrical Engineering}}
	{\textbf{September 2012 - May 2016}}

Minors: Product Design, Math

University of Minnesota-Twin Cities

College of Science and Engineering

Honors: Summa Cum Laude with High Distinction

%% --- SKILLS Section --- %%
\subsection*{\sansserif SKILLS:}
Python with TensorFlow and Keras experience, OpenCV, MATLAB, ROS, 
	Microsoft Office Suite, \LaTeX, Linux/Ubuntu, git, svn

%% --- RESEARCH INTERESTS --- %%
\begin{comment}
\subsection*{\sansserif RESEARCH INTERESTS:}
\begin{itemize}
	\item Efficient sensor fusion, particularly for autonomous vehicles
	\item Real-time video tracking and segmentation
	\item Intuitive exploitation of sequential nature of videos for perception tasks
\end{itemize}
\end{comment}


%% --- RESEARCH EXPERIENCE --- %%
\subsection*{\sansserif RESEARCH EXPERIENCE:}

%% Radar-video fusion
\headerrow
    {\uline{RADAR-Video Fusion} \textit{(UIUC)}}
    {November 2017 - December 2019}
    \begin{itemize}
        \item
        Working with Professor Minh Do to develop fusion algorithms to generate high resolution depth maps from RADAR, video, and egomotion data for autonomous vehicle applications
	\item 
		Investigated Frequency-Modulated Continuous-Wave (FMCW) radar
		    chirp parameters using Texas Instruments mmWave platform for
		    raw ADC data collection 
    \end{itemize}

\headerrow
	{\uline{Small Target Detection and Background Estimation} \textit{(UIUC,
	Sandia National Laboratories)}}
	{October 2018 - December 2019}
	\begin{itemize}
		\item
			Worked with Professor Minh Do and collaborators at
			Sandia National Laboratories to investigate unsupervised
			background estimation methods for video sequences with
			small, low-resolution targets (e.g. satellite video)
	\end{itemize}

%% Video segmentation
\headerrow
	{\uline{Video Segmentation} \textit{(UIUC)}}
	{January 2017 - September 2017}
	\begin{itemize}
		\item
		Working with Professor Minh Do to develop online, unsupervised video object segmentation algorithm for the DAVIS dataset using clustering of dense optical flow (partnership with Sandia National Laboratories)	
	\end{itemize}

%% Massive MIMO (undergrad honors thesis)
\headerrow
	{\uline{VLSI Design and Evaluation of a Massive MIMO Detection Algorithm} \textit{(UMN)}}
	{September 2015 - May 2016}
	\begin{itemize}
		\item
		Worked with Professor Gerald Sobelman on surveying and evaluating proposed massive MIMO detection algorithms and implementing one in Verilog for VLSI (project for Honors Thesis) 
	\end{itemize}


%% --- PRIOR WORK EXPERIENCE --- %%
\subsection*{\sansserif PRIOR WORK EXPERIENCE}

%% Grad
%% RA
\headerrow
	{\uline{Research Assistant} \textit{(UIUC)}}
	{August 2017 - December 2019}
	\begin{itemize}
		\item Developed online algorithm for unsupervised video object segmentation based on clustering of dense optical flow
		\item Investigated real-time trackers based on discriminative correlation filters
        \item Developed RADAR-video fusion algorithms to generate high
resolution depth maps using electronically scanning RADAR (ESR) sensors found on
modern cars paired with video
	\end{itemize}
	%\newline %comment out if including itemize info
	
%% SNL
\headerrow
	{\uline{Graduate Student Intern} \textit{(Sandia National Laboratories)}}
	{September 2018 - December 2019}
		\begin{itemize}
			\item Member of the Sensor Specific Processing team
				working on the transient detection pipeline
			\item Developed transient event data simulator with
				ground truth for multiple profiles
			\item Developed preliminary pipeline precision-recall
				testing with synthesized data for cloud-based
				detection pipeline
			\item Investigated unsupervised background estimation
				for small, moving targets
		\end{itemize}
	%\newline

%% SwRI
\headerrow
	{\uline{Graduate Student in Critical Systems Department} \textit{(Southwest Research Institute)}}
	{May 2017 - August 2018}
	\begin{itemize}%[leftmargin=*]
		\item[] Summer 2017
			\begin{itemize}
				\item Trained and tested convolutional neural networks for methane leak detection and segmentation to be implemented on an NVIDIA Tegra embedded platform
				\item Implemented a parallelized data simulator from MATLAB code in Python to generate training and testing datasets
				\item Trained several shallow, efficient networks for preliminary investigation for an internal research project
			\end{itemize}
		\item[] Summer 2018
			\begin{itemize} 
				\item Investigated the use of computer vision features with convolutional neural networks for gas leak flow rate quantification
				\item Developed algorithms for predicting anomalous events in high dimensional, multi-sensor time series data 
			\end{itemize}
	\end{itemize}
	%\newline

%% ECE 110 TAing
\headerrow
	{\uline{Introduction to Electronics Lab Teaching Assistant} \textit{(UIUC)}}
	{August 2016 - December 2017}
	\begin{itemize}
		\item Led basic electronics lab sessions weekly and graded lab reports weekly
	\end{itemize}
	%\newline %comment out if including itemize info

%% Undergrad
%% EE 2001 LA
\headerrow
	{\uline{Introduction to Circuits and Electronics Lecture Assistant} \textit{(UMN)}}
	{January 2016 - May 2016}
	\begin{itemize}
		\item Held weekly office hours to help students learn the basics of circuits and electronics and graded exams
	\end{itemize}


%% EE 2301 LA
\headerrow
	{\uline{Introduction to Digital System Design Lecture Assistant} \textit{(UMN)}}
	{September 2015 - December 2015}
	\begin{itemize}
		\item Held weekly office hours to assit students with learning the basics of digital system design and graded homeworks and exams%, including BOolean algebra, Karnaugh maps, Quine-McCluskey method, sequential logic, and other essential digital design techniques
	\end{itemize}

%% Toro Internship
\headerrow
	{\uline{Commercial Electrical Engineering Intern} \textit{(The Toro Company)}}
	{May 2015 - August 2015}
	\begin{itemize}
		\item Investigated warranty claims on field-return control boards to determine possible failure modes and the statistics of the failure rates
		\item Designed prototype wireless test platforms, including a GPS and Bluetooth-based asset tracker and a wireless joystick control platform
		\item Investigated the internal components and safety ratings for high-current, three-phase AC motor contorllers for comparison against the various vendor ratings, to provide a final recommendation for use in the fully electric mid-duty Toro Workman
	\end{itemize}

%% Taylor Center Tutor 
\headerrow
	{\uline{Taylor Center Tutor} \textit{(UMN)}}
	{September 2013 - December 2014}
	\begin{itemize}	
		\item Tutored freshmen in Physics I \& II, Calculus I, II, \& III, and Linear Algebra at Frontier Residence Hall
	\end{itemize}


%% --- AWARDS / FELLOWSHIPS --- %%
\subsection*{\sansserif AWARDS}

%% Qualcomm Innovation Fellowship
\begin{itemize}
	\item Qualcomm Innovation Fellowship 2018 Finalist
\end{itemize}


%% --- ACTIVITIES and LEADERSHIP EXPERIENCES --- %%
\subsection*{\sansserif{ACTIVITIES AND LEADERSHIP EXPERIENCES:}}

%% HKN - UMN
{\uline{IEEE - Eta Kappa Nu} \textit{(UMN)}

\headerrow
	{\quad\textit{President, Omicron Chapter}}
	{May 2015 - May 2016}
	\begin{itemize}
		\item Organized tutoring schedule for members and worked wiht the Electrical and Computer Engineering Department to provide funding for members meeting enough tutoring hours per semester
		\item Extended invitiations to new members and organized induction ceremony and dineer
	\end{itemize}

\headerrow
	{\quad\textit{Member, Omicron Chapter}}
	{December 2015 - May 2016}
	\begin{itemize}
		\item Tutored lower level electrical engineering courses weekly as a part of the IEEE Honors society
	\end{itemize}

%% Tau Beta Pi - UMN
{\uline{Tau Beta Pi} \textit{(UMN)}

\headerrow
	{\quad\textit{Member, Minnesota Alpha Chapter}}
	{April 2013 - May 2016}
	\begin{itemize}
		\item Developed team chartering skills through Engineering Futures sessions and developed other engineering skills as a memer of the oldest engineering honors society in the nation
	\end{itemize}

%% Marching Band - UMN
{\uline{Marching Band} \textit{(UMN)}}

\headerrow
	{\quad\textit{Leader, University of Minnesota Marching Band}}
	{May 2014 - December 2015}
	\begin{itemize}
		\item Taught marching fundamentals and expectations to the 2014 and 2015 rookie clases and led weekly small group sectionals
	\end{itemize}

\headerrow
	{\quad\textit{Member, University of Minnesota Marching Band}}
	{August 2012 - December 2015}
	\begin{itemize}
		\item Rehearsed for 500+ hours each fall semester to build strong discipline, personal motivation, teamwork, self-awareness, and musicianship to foster a strong sense of Gopher Pride at the University of Minnesota
	\end{itemize}

%% Pep Band - UMN
{\uline{Pep Band} \textit{(UMN)}}

\headerrow
	{\quad\textit{Member, University of Minnesota Gold Pep Band}}
	{September 2013 - May 2016}
	\begin{itemize}
		\item Performed at men's basketball games and other sporting events, including several NCAA championship games (women's hockey and women's volleyball) and the Big Ten men's basketball tournament
	\end{itemize}

\headerrow
	{\quad\textit{Member, University of Minnesota Gopher Pep Band}}
	{September 2012 - May 2013}
	\begin{itemize}
		\item Perfored at women's hocky, women's basketball, and women's volleyball sporting events, including the women's hockey NCAA and WCHA championships
	\end{itemize}

%% Colleges Against Cancer - UMN
{\uline{Colleges Against Cancer} \textit{(UMN)}}

\headerrow
	{\quad\textit{Director of Event Development}}
	{May 2014 - May 2016}
	\begin{itemize}
		\item Orgainzed the 2015 and 2016 Relay for Life fundraisers at the University of Minnesota, with each having over 1,000 participants attending
	\end{itemize}

\headerrow
	{\quad\textit{Chair of Event Logistics}}
	{October 2013 - May 2014}
	\begin{itemize}
		\item Worked with the Director of Event Development to organize and coordinate the 2014 Relay for Life event at the University of Minnesota
	\end{itemize}


%% --- HONORS & AWARDS --- %%
\begin{comment}
\subsection*{HONORS \& AWARDS:}

%% Undergrad
{\uline{Undergraduate Awards}}
\begin{itemize}
	\item National Merit Scholarship
	\item Gold Scholarship, University of Minnesota
	\item Bentson Family Scholarship, University of Minnesota
	\item Hartig Fund Scholarship, University of Minnesota
	\item Red McCleod Scholarship, University of Minnesota
	\item Undergraduate Research Award, University of Minnesota
	\item Class AA "Triple A Award" Winner, Minnesota State High School League
	\item Thomas J. Watson Scholarship, IBM
	\item University Honors Program Honors Student
\end{itemize}
\end{comment}

\end{document}
